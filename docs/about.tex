h2. My Green Town

One of today's greatest problems are pollution and ecology. We consider that children should be educated in the spirit of leading an ecological lifestyle. The problem is that in some countries, ecology is not a priority in education and there are no interactive and entertaining methods of obtaining information about this topic.

Our solution to this issue comes in the form of a small web-based game. In our game, players (children and grown-ups too) can build their own ecological town, while they must take care of many factors which affect ecology and the efficiency of a town.

Our goal is to teach players about ecology in an entertaining way. Also, we provide a fun, non-violent and educative game for everyone to play.

We consider that we have achieved this goal with the game we have developed: in our game, each player has to build its own town from scratch. This way, players get to know each building closely and learn about their disadvantages and advantages. Basically, at each step a player will have 2 options: he can either build a cheaper, polluting building or a modern, ecological version, which may be more expansive. Of course, the ultimate goal is to make the town as ecological as possible. You are limited by a budget, so building ecological buildings is not always a viable options. This means that the players will have to find a balance between ecological and non-ecological buildings. To do this, players will have to know what each building does and why and when is it good to build one of them. So, our game puts players in front of a puzzle which they'll have to solve.

One of the best aspects of the game is the user interface: we have designed it to be as easy to use as possible, so users can focus on the game itself, finding controls for everything easily. This way, the statistics, tasks, profile and information about the buildings are just one click away.
To aid players, we have developed a task system: players can fulfill task to gain rewards and aid them to build up their town. The tasks are a great way for new players to learn the basics of building an ecological town.

Finally, we hope that you will enjoy playing our game and you will considering using ecological alternatives in the future!