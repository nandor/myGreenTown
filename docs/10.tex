h1. What are solar panels?

h2. LEARN MORE ABOUT SOLAR PANELS

Solar Panels are a form of active solar power, a term that describes how solar panels make use of the sun's energy: solar panels harvest sunlight and actively convert it to electricity. Solar Cells, or photovoltaic cells, are arranged in a grid-like pattern on the surface of the solar panel. These solar voltaic cells collect sunlight during the daylight hours and covert it into electricity. 

h2. WHAT ARE SOLAR PANELS MADE OF?

Solar panels are typically constructed with cystalline silicon, which is used in other industries (such as the microprocessor industry), and the more expensive gallium arsenide, which is produced exclusively for use in photovoltaic (solar) cells. 
Other, more efficient solar panels are assembled by depositing amorphous silicon alloy in a continuous roll-to-roll process. The solar cells created from this process are called Amorphous Silicon Solar Cells, or A-si. Solar Panels constructed using amorphous silicon technology are more durable, efficient, and thinner than their crystalline counterparts. 
For very important solar projects, such as space probes that have to rely on solar energy, very-high efficiency solar cells are constructed from gallium arsenide by a process called molecular beam epitaxy. Solar cells constructed by this process have several p-n junction diodes, each designed to be maximally efficient at absorbing a given part of the solar spectrum. This solar panels are much more efficient than conventional types, but the process and materials involved make them far too expensive for everyday applications. 
The newest solar panels function on the molecular or quantum level, and represent an exciting new technology coming into play. These solar panels are created by implanting carbon nanotubes or quantum dots into a treated plastic. Unlike silicon-based solar panels, these solar panels do not have to be constructed in a clean room, and therefore production costs are somewhat dimished. 
For continued instruction in the manufacture of solar panels, see How solar panels are made. 
The practical applications of solar panels constructed from plastics are staggering - they could be overlayed onto a laptop screen to provide continous power, or provide supplemental power to any number of outdoor appliances. The primary hurdle for this new technology is efficiency, and these 'plastic' solar panels have an operational efficiency of about .11% of their silicon-based counterparts. The only short-term solution to this energy problem is for these plastic solar panels to generate electricity from light outside the visible spectrum of light. Some highly-experimental plastic solar panels have been made to absorb infrared energy, and if a solar panel is made that can absorb both infrared energy and light from the visible spectrum, the operational efficiency could increase up to thirty percent. 

h2. HOW MUCH POWER DO SOLAR PANELS PRODUCE?

In direct sunlight at the surface of the equator, a maximally efficient photovoltaic cell about 1/5m in diameter creates a current of approximately 2 amps at 2 volts, however, due to the Earth's atmospheric interference, terran solar panels will never perform as well as solar panels exposed directly to the sun's rays. (see space-based solar power.) 
Years of overheating and physical wear can, however, reduce the operation efficiency of the photovoltaic unit. Solar cells become less efficient over time, and excess energy is released into its thermally conductive substrate as infrared heat. 
The amount of power solar panels produce is influenced by the quality of the solar panel, the materials and technology used in making the solar panel, and the amount of time the solar panel has been in use. When purchasing solar panels, it is therefore wise to look beyond size and look at the dollars/watt ratio. 

h2. WHERE DOES SOLAR ENERGY COME FROM?

Solar energy originates in the depths of our sun. The sun endures a continuous stream of thermonuclear explosions as hydrogen atoms are fused into helium atoms. We encounter the resultant energy as radiation that strikes the surface of the earth. Solar panels convert this solar radiation into useful electrical energy and store them in batteries for our use. Enough solar radiation strikes the earth every day to meet earth's energy needs for an entire year. Solar panels help us harvest this energy and convert it into usable energy to meet the everyday needs of modern life.
How do solar panels work?

h2. LEARN HOW SOLAR PANELS CONVERT SUNLIGHT INTO ELECTRICITY

Solar panels collect solar radiation from the sun and actively convert that energy to electricity. Solar panels are comprised of several individual solar cells. These solar cells function similarly to large semiconductors and utilize a large-area p-n junction diode. When the solar cells are exposed to sunlight, the p-n junction diodes convert the energy from sunlight into usable electrical energy. The energy generated from photons striking the surface of the solar panel allows electrons to be knocked out of their orbits and released, and electric fields in the solar cells pull these free electrons in a directional current, from which metal contacts in the solar cell can generate electricity. The more solar cells in a solar panel and the higher the quality of the solar cells, the more total electrical output the solar panel can produce. The conversion of sunlight to usable electrical energy has been dubbed the Photovoltaic Effect. 
The photovoltaic effect arises from the properties of the p-n junction diode, as such there are no moving parts in a solar panel. 

h2. SOLAR INSOLATION AND SOLAR PANEL EFFICIENCY

Solar Insolation is a measure of how much solar radiation a given solar panel or surface recieves. The greater the insolation, the more solar energy can be converted to electricity by the solar panel. Click to learn more about solar insolation. 
Other factors that affect the output of solar panels are weather conditions, shade caused by obstructions to direct sunlight, and the angle and position at which the solar panel is installed. Solar panels function the best when placed in direct sunlight, away from obstructions that might cast shade, and in areas with high regional solar insolation ratings. 
Solar panel efficiency can be optimized by using dynamic mounts that follow the position of the sun in the sky and rotate the solar panel to get the maximum amount of direct exposure during the day as possible. For more information on solar panel efficiency through the use of mounts, see our section on solar panel mounts and accessories.
Why are solar panels necessary?

h2. DISCOVER WHY SOLAR PANELS ARE A VITAL PART OF FUFILLING OUR ENERGY NEEDS

Although there are numerous other methods of generating electricity, solar panels have a number of considerable advantages for both the consumer, the producer, and the environment. 

h2. USES OF SOLAR PANELS

Solar panels are used to power all sorts of electronic equipment, from solar-powered handheld calculators that will function as long as sunlight is available, to remote solar-powered sensor arrays in bouys, and even some experimental vehicles and boats. Solar panels are also placed on outdoor lighting structures - the solar cell is charged during daylight hours, and at night, we get free electricity to keep our streets well-lit and secure. Solar panels are used extensively on satellites, where array of solar cells provide reliable power for the satellite's electrical systems.
Solar Power plants, which are large collections of solar panels arranged to generate commercial electricity, are becoming more and more frequent these days. While still between 2-5 times as expensive to produce as electricity from fossil fuels, electricity generated from solar panels is free, nearly infinitely abundant, and non-polluting. Many environmentally-minded communities across America have set up solar power stations to help generate private or commercial solar energy. 
Powering homes with solar power has also been a major part of the solar revolution the last two decades have seen. Solar panels can be placed on the roof of homes, businesses, or remote research stations, and can be used independent of or in conjunction with the local power grid. 

h2. ADVANTAGES OF SOLAR PANELS

Solar panels are clean - while generating electricity from sunlight, solar panels produce virtually no pollution, whereas burning fossil fuels releases large quantities of toxic gases into the atmosphere. 
For the consumer, solar panels can free the individual from reliance on the power grid and the monopolistic energy supplier. Once you make the initial investment in hardware, you will have free electricity for years to come. 
Fossil Fuels are limited - Although fossil fuel reserves are expected to run dry within the next century, solar power is clean, abundant, and will remain a renewable resource that can meet all of Earth's energy needs for billions of years to come. 

h2. PORTABLE SOLAR PANELS

Portability is a major advantage of the common small watting rating solar panel, which can be used in numerous electronic and handheld devices when off the electricity grid, or when lugging around a generator would be impractical or wasteful. Portable or device-mounted solar panels can power solar-powered calculators, laptops, and even small motorized vehicles. The applications are limitless, but as you will see, the costs of using solar panels are not to be trivialized. 

h2. DRAWBACKS OF SOLAR PANELS

Admittedly, while solar power is certainly much cleaner than the burning of fossil fuels, and moderately cleaner than the production of nuclear power, solar panels are very pricey and in many years demand for solar panels exceeds supply. When we ask ourselves - why are solar panels necessary, we must consider the costs of production as well as the costs of using much more harmful means of producing electricity. Solar Panels also require more square yardage per kilowatt for the power-generating facility than fossil fuel power plants or nuclear power. 
How much energy do solar panels produce?

h2. LEARN MORE ABOUT THE PHOTOVOLTAIC CONVERSION OF SUNLIGHT INTO USEFUL ENERGY

As we previously mentioned, solar panels collect solar radiation from the sun and actively convert that energy to electricity. The solar cells on these solar panels make use of the extremely small fraction of the sun's energy that passes through earth's atmosphere and strikes the cells on the solar collector. The efficiency of these solar panels, and the resultant energy produced is dependant on many climatic, geographic, and weather-related factors. Arid climates are ideal for solar panels, and they will produce more energy in areas where they are exposed to direct sunlight under clear skies. But even at optimal efficiency, solar panels only convert a small percentage of the energy that strikes it into usable energy. The efficiency factors is in the teens for most solar cells. Advanced solar cells, like those used on the Voyager spacecraft, have much higher efficiency ratings, but are much too expensive to produce en masse for general purposes. 

h2. ENERGY FROM THE SUN?

Solar panels have the ability to meet all of our energy needs, but at present we only use a tiny fraction of the energy that the sun has to offer. How much energy does the sun produce? How is it produced? And how much of the sun's energy can be theoretically harvested via solar cells? 
Energy from the sun is caused from thermonuclear expolosions deep within the sun. These explosions fuse atoms of hydrogen into atoms of helium. A tremendous amount of energy is released during the thermonuclear reaction and the sun releases that energy as radiation. This radiation travels through space at the speed of light, and solar panels can make practical use of it. Our sun generates an enourmous amount of energy, and potentially, had we the technology to harvest that sunlight with solar arrays across the solar system, we could harvest huge amounts of energy. 
According to our friends at Astronomy Cafe, we calculate the amount of energy given off the sun every hour as: 
"3.8 x 10^33 ergs/sec or 3.8 x 10^26 watts of power, an amount of energy each second equal to 3.8 x 10^26 joules. In one hour, or 3600 seconds, [the Sun] produces 1.4 x 10^31 Joules of energy or 3.8 x 10^23 kilowatt-hours." 
The sun produces more energy every hour than the entire energy needs of human civilization from the beginning of time. Solar panels will help us harvest increasing amounts of this abundance of energy to meet our energy needs in the future. 